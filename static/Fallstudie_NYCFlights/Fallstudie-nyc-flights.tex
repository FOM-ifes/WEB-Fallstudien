\documentclass[]{article}
\usepackage{lmodern}
\usepackage{amssymb,amsmath}
\usepackage{ifxetex,ifluatex}
\usepackage{fixltx2e} % provides \textsubscript
\ifnum 0\ifxetex 1\fi\ifluatex 1\fi=0 % if pdftex
  \usepackage[T1]{fontenc}
  \usepackage[utf8]{inputenc}
\else % if luatex or xelatex
  \ifxetex
    \usepackage{mathspec}
  \else
    \usepackage{fontspec}
  \fi
  \defaultfontfeatures{Ligatures=TeX,Scale=MatchLowercase}
\fi
% use upquote if available, for straight quotes in verbatim environments
\IfFileExists{upquote.sty}{\usepackage{upquote}}{}
% use microtype if available
\IfFileExists{microtype.sty}{%
\usepackage{microtype}
\UseMicrotypeSet[protrusion]{basicmath} % disable protrusion for tt fonts
}{}
\usepackage[margin=1in]{geometry}
\usepackage{hyperref}
\hypersetup{unicode=true,
            pdftitle={NYC Flights},
            pdfauthor={Sebastian Sauer},
            pdfborder={0 0 0},
            breaklinks=true}
\urlstyle{same}  % don't use monospace font for urls
\usepackage{color}
\usepackage{fancyvrb}
\newcommand{\VerbBar}{|}
\newcommand{\VERB}{\Verb[commandchars=\\\{\}]}
\DefineVerbatimEnvironment{Highlighting}{Verbatim}{commandchars=\\\{\}}
% Add ',fontsize=\small' for more characters per line
\usepackage{framed}
\definecolor{shadecolor}{RGB}{248,248,248}
\newenvironment{Shaded}{\begin{snugshade}}{\end{snugshade}}
\newcommand{\AlertTok}[1]{\textcolor[rgb]{0.94,0.16,0.16}{#1}}
\newcommand{\AnnotationTok}[1]{\textcolor[rgb]{0.56,0.35,0.01}{\textbf{\textit{#1}}}}
\newcommand{\AttributeTok}[1]{\textcolor[rgb]{0.77,0.63,0.00}{#1}}
\newcommand{\BaseNTok}[1]{\textcolor[rgb]{0.00,0.00,0.81}{#1}}
\newcommand{\BuiltInTok}[1]{#1}
\newcommand{\CharTok}[1]{\textcolor[rgb]{0.31,0.60,0.02}{#1}}
\newcommand{\CommentTok}[1]{\textcolor[rgb]{0.56,0.35,0.01}{\textit{#1}}}
\newcommand{\CommentVarTok}[1]{\textcolor[rgb]{0.56,0.35,0.01}{\textbf{\textit{#1}}}}
\newcommand{\ConstantTok}[1]{\textcolor[rgb]{0.00,0.00,0.00}{#1}}
\newcommand{\ControlFlowTok}[1]{\textcolor[rgb]{0.13,0.29,0.53}{\textbf{#1}}}
\newcommand{\DataTypeTok}[1]{\textcolor[rgb]{0.13,0.29,0.53}{#1}}
\newcommand{\DecValTok}[1]{\textcolor[rgb]{0.00,0.00,0.81}{#1}}
\newcommand{\DocumentationTok}[1]{\textcolor[rgb]{0.56,0.35,0.01}{\textbf{\textit{#1}}}}
\newcommand{\ErrorTok}[1]{\textcolor[rgb]{0.64,0.00,0.00}{\textbf{#1}}}
\newcommand{\ExtensionTok}[1]{#1}
\newcommand{\FloatTok}[1]{\textcolor[rgb]{0.00,0.00,0.81}{#1}}
\newcommand{\FunctionTok}[1]{\textcolor[rgb]{0.00,0.00,0.00}{#1}}
\newcommand{\ImportTok}[1]{#1}
\newcommand{\InformationTok}[1]{\textcolor[rgb]{0.56,0.35,0.01}{\textbf{\textit{#1}}}}
\newcommand{\KeywordTok}[1]{\textcolor[rgb]{0.13,0.29,0.53}{\textbf{#1}}}
\newcommand{\NormalTok}[1]{#1}
\newcommand{\OperatorTok}[1]{\textcolor[rgb]{0.81,0.36,0.00}{\textbf{#1}}}
\newcommand{\OtherTok}[1]{\textcolor[rgb]{0.56,0.35,0.01}{#1}}
\newcommand{\PreprocessorTok}[1]{\textcolor[rgb]{0.56,0.35,0.01}{\textit{#1}}}
\newcommand{\RegionMarkerTok}[1]{#1}
\newcommand{\SpecialCharTok}[1]{\textcolor[rgb]{0.00,0.00,0.00}{#1}}
\newcommand{\SpecialStringTok}[1]{\textcolor[rgb]{0.31,0.60,0.02}{#1}}
\newcommand{\StringTok}[1]{\textcolor[rgb]{0.31,0.60,0.02}{#1}}
\newcommand{\VariableTok}[1]{\textcolor[rgb]{0.00,0.00,0.00}{#1}}
\newcommand{\VerbatimStringTok}[1]{\textcolor[rgb]{0.31,0.60,0.02}{#1}}
\newcommand{\WarningTok}[1]{\textcolor[rgb]{0.56,0.35,0.01}{\textbf{\textit{#1}}}}
\usepackage{graphicx,grffile}
\makeatletter
\def\maxwidth{\ifdim\Gin@nat@width>\linewidth\linewidth\else\Gin@nat@width\fi}
\def\maxheight{\ifdim\Gin@nat@height>\textheight\textheight\else\Gin@nat@height\fi}
\makeatother
% Scale images if necessary, so that they will not overflow the page
% margins by default, and it is still possible to overwrite the defaults
% using explicit options in \includegraphics[width, height, ...]{}
\setkeys{Gin}{width=\maxwidth,height=\maxheight,keepaspectratio}
\IfFileExists{parskip.sty}{%
\usepackage{parskip}
}{% else
\setlength{\parindent}{0pt}
\setlength{\parskip}{6pt plus 2pt minus 1pt}
}
\setlength{\emergencystretch}{3em}  % prevent overfull lines
\providecommand{\tightlist}{%
  \setlength{\itemsep}{0pt}\setlength{\parskip}{0pt}}
\setcounter{secnumdepth}{0}
% Redefines (sub)paragraphs to behave more like sections
\ifx\paragraph\undefined\else
\let\oldparagraph\paragraph
\renewcommand{\paragraph}[1]{\oldparagraph{#1}\mbox{}}
\fi
\ifx\subparagraph\undefined\else
\let\oldsubparagraph\subparagraph
\renewcommand{\subparagraph}[1]{\oldsubparagraph{#1}\mbox{}}
\fi

%%% Use protect on footnotes to avoid problems with footnotes in titles
\let\rmarkdownfootnote\footnote%
\def\footnote{\protect\rmarkdownfootnote}

%%% Change title format to be more compact
\usepackage{titling}

% Create subtitle command for use in maketitle
\providecommand{\subtitle}[1]{
  \posttitle{
    \begin{center}\large#1\end{center}
    }
}

\setlength{\droptitle}{-2em}

  \title{NYC Flights}
    \pretitle{\vspace{\droptitle}\centering\huge}
  \posttitle{\par}
    \author{Sebastian Sauer}
    \preauthor{\centering\large\emph}
  \postauthor{\par}
      \predate{\centering\large\emph}
  \postdate{\par}
    \date{2019-08-01}


\begin{document}
\maketitle

{
\setcounter{tocdepth}{2}
\tableofcontents
}
\hypertarget{fallstudie-nyc-flights}{%
\section{Fallstudie NYC Flights}\label{fallstudie-nyc-flights}}

Aufgabe: Vorhersage von Verspätungen der Flüge von den Flughäfen von New
York City im Jahr 2013.

\hypertarget{vorbereitung}{%
\subsection{Vorbereitung}\label{vorbereitung}}

Pakete laden:

\begin{Shaded}
\begin{Highlighting}[]
\KeywordTok{library}\NormalTok{(mosaic)}
\KeywordTok{library}\NormalTok{(tidyverse)}
\KeywordTok{library}\NormalTok{(lubridate)}
\KeywordTok{library}\NormalTok{(corrr)}
\KeywordTok{library}\NormalTok{(caret)}
\KeywordTok{library}\NormalTok{(doMC)}
\KeywordTok{library}\NormalTok{(ranger)}
\KeywordTok{library}\NormalTok{(sjmisc)}
\end{Highlighting}
\end{Shaded}

Daten laden:

\begin{Shaded}
\begin{Highlighting}[]
\KeywordTok{library}\NormalTok{(nycflights13)}
\KeywordTok{data}\NormalTok{(flights)}
\KeywordTok{glimpse}\NormalTok{(flights)}
\CommentTok{#> Observations: 336,776}
\CommentTok{#> Variables: 19}
\CommentTok{#> $ year           <int> 2013, 2013, 2013, 2013, 2013, 2013, 2013, 2013,...}
\CommentTok{#> $ month          <int> 1, 1, 1, 1, 1, 1, 1, 1, 1, 1, 1, 1, 1, 1, 1, 1,...}
\CommentTok{#> $ day            <int> 1, 1, 1, 1, 1, 1, 1, 1, 1, 1, 1, 1, 1, 1, 1, 1,...}
\CommentTok{#> $ dep_time       <int> 517, 533, 542, 544, 554, 554, 555, 557, 557, 55...}
\CommentTok{#> $ sched_dep_time <int> 515, 529, 540, 545, 600, 558, 600, 600, 600, 60...}
\CommentTok{#> $ dep_delay      <dbl> 2, 4, 2, -1, -6, -4, -5, -3, -3, -2, -2, -2, -2...}
\CommentTok{#> $ arr_time       <int> 830, 850, 923, 1004, 812, 740, 913, 709, 838, 7...}
\CommentTok{#> $ sched_arr_time <int> 819, 830, 850, 1022, 837, 728, 854, 723, 846, 7...}
\CommentTok{#> $ arr_delay      <dbl> 11, 20, 33, -18, -25, 12, 19, -14, -8, 8, -2, -...}
\CommentTok{#> $ carrier        <chr> "UA", "UA", "AA", "B6", "DL", "UA", "B6", "EV",...}
\CommentTok{#> $ flight         <int> 1545, 1714, 1141, 725, 461, 1696, 507, 5708, 79...}
\CommentTok{#> $ tailnum        <chr> "N14228", "N24211", "N619AA", "N804JB", "N668DN...}
\CommentTok{#> $ origin         <chr> "EWR", "LGA", "JFK", "JFK", "LGA", "EWR", "EWR"...}
\CommentTok{#> $ dest           <chr> "IAH", "IAH", "MIA", "BQN", "ATL", "ORD", "FLL"...}
\CommentTok{#> $ air_time       <dbl> 227, 227, 160, 183, 116, 150, 158, 53, 140, 138...}
\CommentTok{#> $ distance       <dbl> 1400, 1416, 1089, 1576, 762, 719, 1065, 229, 94...}
\CommentTok{#> $ hour           <dbl> 5, 5, 5, 5, 6, 5, 6, 6, 6, 6, 6, 6, 6, 6, 6, 5,...}
\CommentTok{#> $ minute         <dbl> 15, 29, 40, 45, 0, 58, 0, 0, 0, 0, 0, 0, 0, 0, ...}
\CommentTok{#> $ time_hour      <dttm> 2013-01-01 05:00:00, 2013-01-01 05:00:00, 2013...}
\end{Highlighting}
\end{Shaded}

\hypertarget{explorative-datenanalyse}{%
\subsection{Explorative Datenanalyse}\label{explorative-datenanalyse}}

\hypertarget{wie-ist-die-verspatung-verteilt}{%
\subsubsection{Wie ist die Verspätung
verteilt?}\label{wie-ist-die-verspatung-verteilt}}

Es gibt zwei Variablen, die Verspätung anzeigen: \texttt{arr\_delay}
(Ankunft) und \texttt{dep\_delay}.

\begin{Shaded}
\begin{Highlighting}[]
\KeywordTok{favstats}\NormalTok{(arr_delay }\OperatorTok{~}\StringTok{ }\DecValTok{1}\NormalTok{, }\DataTypeTok{data =}\NormalTok{ flights)}
\CommentTok{#>   1 min  Q1 median Q3  max     mean       sd      n missing}
\CommentTok{#> 1 1 -86 -17     -5 14 1272 6.895377 44.63329 327346    9430}
\KeywordTok{favstats}\NormalTok{(dep_delay }\OperatorTok{~}\StringTok{ }\DecValTok{1}\NormalTok{, }\DataTypeTok{data =}\NormalTok{ flights)}
\CommentTok{#>   1 min Q1 median Q3  max     mean       sd      n missing}
\CommentTok{#> 1 1 -43 -5     -2 11 1301 12.63907 40.21006 328521    8255}
\end{Highlighting}
\end{Shaded}

Nehmen wir \texttt{arr\_delay}, da die Streuung in dieser Variable höher
ist.

\hypertarget{vertiefung}{%
\subsubsection{VERTIEFUNG}\label{vertiefung}}

Möchte man einen Befehl auf mehrere Spalten anwenden, so kann man dafür
den Befehl \texttt{map()} verwendet. \texttt{map()} führt ein Befehl auf
jede Spalte eines Dataframes aus. Damit man da Ergebnis in Form eines
Dataframes (Tabelle) bekommt, fügt man \texttt{\_df} an \texttt{map()}
an:

\begin{Shaded}
\begin{Highlighting}[]
\NormalTok{flights }\OperatorTok\StringTok{ }
\StringTok{  }\KeywordTok{select}\NormalTok{(arr_delay, dep_delay) }\OperatorTok\StringTok{ }
\StringTok{  }\KeywordTok{map_df}\NormalTok{(favstats)}
\CommentTok{#>   min  Q1 median Q3  max      mean       sd      n missing}
\CommentTok{#> 1 -86 -17     -5 14 1272  6.895377 44.63329 327346    9430}
\CommentTok{#> 2 -43  -5     -2 11 1301 12.639070 40.21006 328521    8255}
\end{Highlighting}
\end{Shaded}

Für den IQR:

\begin{Shaded}
\begin{Highlighting}[]
\NormalTok{flights }\OperatorTok\StringTok{ }
\StringTok{  }\KeywordTok{select}\NormalTok{(arr_delay, dep_delay) }\OperatorTok\StringTok{ }
\StringTok{  }\KeywordTok{drop_na}\NormalTok{() }\OperatorTok\StringTok{ }
\StringTok{  }\KeywordTok{map_df}\NormalTok{(iqr)}
\CommentTok{#> # A tibble: 1 x 2}
\CommentTok{#>   arr_delay dep_delay}
\CommentTok{#>       <dbl>     <dbl>}
\CommentTok{#> 1        31        16}
\end{Highlighting}
\end{Shaded}

\hypertarget{visualisierung-der-verteilung}{%
\subsubsection{Visualisierung der
Verteilung}\label{visualisierung-der-verteilung}}

\begin{Shaded}
\begin{Highlighting}[]
\KeywordTok{gf_histogram}\NormalTok{( }\OperatorTok{~}\StringTok{ }\NormalTok{arr_delay, }\DataTypeTok{data =}\NormalTok{ flights)}
\KeywordTok{gf_violin}\NormalTok{(arr_delay }\OperatorTok{~}\StringTok{ }\DecValTok{1}\NormalTok{, }\DataTypeTok{data =}\NormalTok{ flights)}
\end{Highlighting}
\end{Shaded}

\begin{center}\includegraphics[width=0.7\linewidth]{Fallstudie-nyc-flights_files/figure-latex/unnamed-chunk-7-1} \includegraphics[width=0.7\linewidth]{Fallstudie-nyc-flights_files/figure-latex/unnamed-chunk-7-2} \end{center}

Aufgrund des langen rechten Randbereichs (hohe Verspätungswerte) ist das
Diagramm nicht sher hilfreich.

Begrenzen wir uns besser auf den ``inneren'' Teil der Flüge (was die
Verspätung betrifft).

\begin{Shaded}
\begin{Highlighting}[]
\NormalTok{flights }\OperatorTok\StringTok{ }
\StringTok{  }\KeywordTok{filter}\NormalTok{(arr_delay }\OperatorTok{<}\StringTok{ }\DecValTok{120}\NormalTok{) }\OperatorTok\StringTok{ }
\StringTok{  }\KeywordTok{gf_violin}\NormalTok{(arr_delay }\OperatorTok{~}\StringTok{ }\DecValTok{1}\NormalTok{, }\DataTypeTok{data =}\NormalTok{ .)}
\end{Highlighting}
\end{Shaded}

\begin{center}\includegraphics[width=0.7\linewidth]{Fallstudie-nyc-flights_files/figure-latex/unnamed-chunk-8-1} \end{center}

\hypertarget{saisonale-effekte}{%
\subsubsection{Saisonale Effekte}\label{saisonale-effekte}}

Es gibt sehr viele potenzielle Ursachen für die Verspätung eines
Flugzeugs bzw. eines Flugs. Zu einigen Kandidaten liegen uns Daten vor.
Eine naheliegende (obwohl nicht tiefer theoretisch fundierte) Annahme
ist, dass es saisonale Einflüsse auf die Verspätung gibt. So könnte
Schnee im Winter oder Weihnachtsstress zum Jahreswechsel für Verspätung
sorgen. Am Wochenende sind die Menschen entspannter und es wird weniger
gereist. Daher könnte es Samstags und Sonntags zu weniger Verspätung
kommen.

\hypertarget{nach-jahreszeiten}{%
\paragraph{Nach Jahreszeiten}\label{nach-jahreszeiten}}

Berechnen wir die Jahreszeiten:

\begin{Shaded}
\begin{Highlighting}[]
\NormalTok{flights2 <-}\StringTok{ }\NormalTok{flights }\OperatorTok\StringTok{ }
\StringTok{  }\KeywordTok{mutate}\NormalTok{(}\DataTypeTok{season =} \KeywordTok{case_when}\NormalTok{(}
\NormalTok{    month }\OperatorTok\StringTok{ }\KeywordTok{c}\NormalTok{(}\DecValTok{11}\NormalTok{, }\DecValTok{12}\NormalTok{, }\DecValTok{1}\NormalTok{, }\DecValTok{2}\NormalTok{, }\DecValTok{3}\NormalTok{) }\OperatorTok{~}\StringTok{ "winter"}\NormalTok{,}
\NormalTok{    month }\OperatorTok\StringTok{ }\KeywordTok{c}\NormalTok{(}\DecValTok{6}\NormalTok{,}\DecValTok{7}\NormalTok{,}\DecValTok{9}\NormalTok{) }\OperatorTok{~}\StringTok{ "summer"}\NormalTok{,}
\NormalTok{    month }\OperatorTok\StringTok{ }\KeywordTok{c}\NormalTok{(}\DecValTok{4}\NormalTok{, }\DecValTok{5}\NormalTok{) }\OperatorTok{~}\StringTok{ "spring"}\NormalTok{,}
    \OtherTok{TRUE} \OperatorTok{~}\StringTok{ "autumn"}
\NormalTok{  ))}
\end{Highlighting}
\end{Shaded}

Verspätungen nach Jahreszeiten:

\begin{Shaded}
\begin{Highlighting}[]
\KeywordTok{favstats}\NormalTok{(arr_delay }\OperatorTok{~}\StringTok{ }\NormalTok{season, }\DataTypeTok{data =}\NormalTok{ flights2)}
\CommentTok{#>   season min  Q1 median Q3  max     mean       sd      n missing}
\CommentTok{#> 1 autumn -68 -18     -6 10  688 2.944260 38.08791  57374     842}
\CommentTok{#> 2 spring -86 -17     -5 15  931 7.310027 46.03570  55692    1434}
\CommentTok{#> 3 summer -68 -18     -6 16 1127 9.838901 52.59895  82378    2864}
\CommentTok{#> 4 winter -70 -15     -3 14 1272 6.600590 40.96247 131902    4290}
\end{Highlighting}
\end{Shaded}

Im Sommer ist die Verspätung am höchsten. Vielleicht ist es besser,
gleich auf Monate hin zu untersuchen:

\begin{Shaded}
\begin{Highlighting}[]
\KeywordTok{favstats}\NormalTok{(arr_delay }\OperatorTok{~}\StringTok{ }\NormalTok{month, }\DataTypeTok{data =}\NormalTok{ flights2)}
\CommentTok{#>    month min  Q1 median Q3  max       mean       sd     n missing}
\CommentTok{#> 1      1 -70 -15     -3 13 1272  6.1299720 40.42390 26398     606}
\CommentTok{#> 2      2 -70 -15     -3 13  834  5.6130194 39.52862 23611    1340}
\CommentTok{#> 3      3 -68 -18     -6 13  915  5.8075765 44.11919 27902     932}
\CommentTok{#> 4      4 -68 -15     -2 19  931 11.1760630 47.49115 27564     766}
\CommentTok{#> 5      5 -86 -20     -8 11  875  3.5215088 44.23761 28128     668}
\CommentTok{#> 6      6 -64 -15     -2 26 1127 16.4813296 56.13087 27075    1168}
\CommentTok{#> 7      7 -66 -16     -2 27  989 16.7113067 57.11709 28293    1132}
\CommentTok{#> 8      8 -68 -18     -5 14  490  6.0406524 42.59514 28756     571}
\CommentTok{#> 9      9 -68 -23    -12  1 1007 -4.0183636 39.71031 27010     564}
\CommentTok{#> 10    10 -61 -18     -7  7  688 -0.1670627 32.64986 28618     271}
\CommentTok{#> 11    11 -67 -16     -6  8  796  0.4613474 31.38741 26971     297}
\CommentTok{#> 12    12 -68 -11      2 25  878 14.8703553 46.13311 27020    1115}
\end{Highlighting}
\end{Shaded}

Tatsächlich ist die Verspätung im Mittelwert am höchsten im Juni und
Juli. Dabei ist zu beachten, dass die \emph{mediane} Verspätung nur im
Dezember positiv ist: Nur im Dezember haben die Flüge in New York im
Median eine Verspätung.

\hypertarget{weihnachten}{%
\paragraph{Weihnachten}\label{weihnachten}}

Liegt es an Weihnachten? Schauen wir uns die Tage im Dezember (und
Januar?) genauer an. Dazu berechnen wir zuerst einen Spalte, die den Tag
(und die Woche) berechnet.

\begin{Shaded}
\begin{Highlighting}[]
\NormalTok{flights3 <-}\StringTok{ }\NormalTok{flights2 }\OperatorTok\StringTok{ }
\StringTok{  }\KeywordTok{mutate}\NormalTok{(}\DataTypeTok{dayinyear =} \KeywordTok{yday}\NormalTok{(time_hour),}
         \DataTypeTok{day_id =} \DecValTok{365}\OperatorTok{-}\NormalTok{(}\DecValTok{365}\OperatorTok{-}\NormalTok{dayinyear),}
         \DataTypeTok{week =} \KeywordTok{week}\NormalTok{(time_hour))}
\end{Highlighting}
\end{Shaded}

\begin{Shaded}
\begin{Highlighting}[]
\NormalTok{flights3 }\OperatorTok\StringTok{ }
\StringTok{  }\KeywordTok{filter}\NormalTok{( (time_hour }\OperatorTok{>}\StringTok{ "2013-11-30 23:59:59"}\NormalTok{) ) }\OperatorTok\StringTok{ }
\StringTok{  }\KeywordTok{group_by}\NormalTok{(dayinyear) }\OperatorTok\StringTok{ }
\StringTok{  }\KeywordTok{summarise}\NormalTok{(}\DataTypeTok{arr_delay =} \KeywordTok{median}\NormalTok{(arr_delay, }\DataTypeTok{na.rm =} \OtherTok{TRUE}\NormalTok{)) }\OperatorTok\StringTok{ }
\StringTok{  }\KeywordTok{gf_line}\NormalTok{(arr_delay }\OperatorTok{~}\StringTok{ }\NormalTok{dayinyear, }\DataTypeTok{data =}\NormalTok{ .)}
\end{Highlighting}
\end{Shaded}

\begin{center}\includegraphics[width=0.7\linewidth]{Fallstudie-nyc-flights_files/figure-latex/unnamed-chunk-13-1} \end{center}

Etwa zwei Wochen vor Jahresende, also noch deutlich vor den Feiertagen,
kommt es zu den Verspätungsspitzen. Ob zu dieser Zeit die meisten
Menschen in den Weihnachtsurlaub fliegen? Insgesamt lässt diese
Betrachtung offenbar keine starken Schlüsse zu.

\hypertarget{wochenende-vs.-werktage}{%
\paragraph{Wochenende vs.~Werktage}\label{wochenende-vs.-werktage}}

Vielleicht sind die Wochentage die entspannten Tage ohne Verspätung?
Schauen wir nach. Man beachte, dass die Woche in Amerika mit Sonntag (1)
beginnt, demzufolge ist der Samstag der 7. Tag.

\begin{Shaded}
\begin{Highlighting}[]
\NormalTok{flights3 }\OperatorTok\StringTok{ }
\StringTok{  }\KeywordTok{mutate}\NormalTok{(}\DataTypeTok{weekend =} \KeywordTok{if_else}\NormalTok{(}\KeywordTok{wday}\NormalTok{(time_hour) }\OperatorTok\StringTok{ }\KeywordTok{c}\NormalTok{(}\DecValTok{1}\NormalTok{,}\DecValTok{7}\NormalTok{), }\OtherTok{TRUE}\NormalTok{, }\OtherTok{FALSE}\NormalTok{)) }\OperatorTok\StringTok{ }
\StringTok{  }\KeywordTok{group_by}\NormalTok{(weekend) }\OperatorTok\StringTok{ }
\StringTok{  }\KeywordTok{summarise}\NormalTok{(}\DataTypeTok{arr_delay_md =} \KeywordTok{median}\NormalTok{(arr_delay, }\DataTypeTok{na.rm =} \OtherTok{TRUE}\NormalTok{))}
\CommentTok{#> # A tibble: 2 x 2}
\CommentTok{#>   weekend arr_delay_md}
\CommentTok{#>   <lgl>          <dbl>}
\CommentTok{#> 1 FALSE             -4}
\CommentTok{#> 2 TRUE              -8}
\end{Highlighting}
\end{Shaded}

Aha, das sind 4 Minuten weniger im Median am Wochenende (im Vergleich zu
werktags). Im Verhältnis zur Streuung von 31 Minuten (IQR) ist das nicht
Nichts, aber auch nicht die Welt.

\hypertarget{verspatung-pro-tag}{%
\paragraph{Verspätung pro Tag}\label{verspatung-pro-tag}}

\begin{Shaded}
\begin{Highlighting}[]
\NormalTok{flights4 <-}\StringTok{ }\NormalTok{flights3 }\OperatorTok\StringTok{ }
\StringTok{  }\KeywordTok{mutate}\NormalTok{(}\DataTypeTok{day_rounded =} \KeywordTok{round_date}\NormalTok{(time_hour, }\StringTok{"day"}\NormalTok{))}

\NormalTok{flights4 }\OperatorTok\StringTok{ }
\StringTok{  }\KeywordTok{group_by}\NormalTok{(day_rounded) }\OperatorTok\StringTok{ }
\StringTok{  }\KeywordTok{summarise}\NormalTok{(}\DataTypeTok{arr_delay_md =} \KeywordTok{median}\NormalTok{(arr_delay, }\DataTypeTok{na.rm =} \OtherTok{TRUE}\NormalTok{)) }\OperatorTok\StringTok{ }
\StringTok{  }\KeywordTok{gf_line}\NormalTok{(arr_delay_md }\OperatorTok{~}\StringTok{ }\NormalTok{day_rounded, }\DataTypeTok{data =}\NormalTok{ .) }\OperatorTok\StringTok{ }
\StringTok{  }\KeywordTok{gf_smooth}\NormalTok{()}
\end{Highlighting}
\end{Shaded}

\begin{center}\includegraphics[width=0.7\linewidth]{Fallstudie-nyc-flights_files/figure-latex/unnamed-chunk-15-1} \end{center}

Die Spitzen sind so nicht direkt erschließbar. Betrachten wir
abschließend die Verspätungen pro Woche.

\begin{Shaded}
\begin{Highlighting}[]
\NormalTok{flights4 }\OperatorTok\StringTok{ }
\StringTok{  }\KeywordTok{mutate}\NormalTok{(}\DataTypeTok{week_rounded =} \KeywordTok{round_date}\NormalTok{(time_hour, }\StringTok{"week"}\NormalTok{)) }\OperatorTok\StringTok{ }
\StringTok{  }\KeywordTok{group_by}\NormalTok{(week_rounded) }\OperatorTok\StringTok{ }
\StringTok{  }\KeywordTok{summarise}\NormalTok{(}\DataTypeTok{arr_delay_md =} \KeywordTok{median}\NormalTok{(arr_delay, }\DataTypeTok{na.rm =} \OtherTok{TRUE}\NormalTok{)) }\OperatorTok\StringTok{ }
\StringTok{  }\KeywordTok{gf_line}\NormalTok{(arr_delay_md }\OperatorTok{~}\StringTok{ }\NormalTok{week_rounded, }\DataTypeTok{data =}\NormalTok{ .) }\OperatorTok\StringTok{ }
\StringTok{  }\KeywordTok{gf_smooth}\NormalTok{()}
\end{Highlighting}
\end{Shaded}

\begin{center}\includegraphics[width=0.7\linewidth]{Fallstudie-nyc-flights_files/figure-latex/unnamed-chunk-16-1} \end{center}

Der Zacken im Juli könnte mit dem Nationalfeierag in den USA
zusammenhängen. Lassen wir diese Untersuchungen an dieser Stelle.

\hypertarget{wetter}{%
\subsubsection{Wetter}\label{wetter}}

Die Wetterdaten sind in einer anderen Tabelle (\texttt{weather}), auch
im Paket \texttt{nycflights13} gespeichert. Über Datum/Zeit können wir
die Wetterdaten mit den Flugdaten zusammenführen. Dabei begnügen wir uns
mit einer tagesgenauen Präzision, da die Wetterdaten nicht jede Stunde
(Minute, Sekunde) abdecken.

\begin{Shaded}
\begin{Highlighting}[]
\KeywordTok{data}\NormalTok{(weather)}
\KeywordTok{glimpse}\NormalTok{(weather)}
\CommentTok{#> Observations: 26,115}
\CommentTok{#> Variables: 15}
\CommentTok{#> $ origin     <chr> "EWR", "EWR", "EWR", "EWR", "EWR", "EWR", "EWR", "E...}
\CommentTok{#> $ year       <dbl> 2013, 2013, 2013, 2013, 2013, 2013, 2013, 2013, 201...}
\CommentTok{#> $ month      <dbl> 1, 1, 1, 1, 1, 1, 1, 1, 1, 1, 1, 1, 1, 1, 1, 1, 1, ...}
\CommentTok{#> $ day        <int> 1, 1, 1, 1, 1, 1, 1, 1, 1, 1, 1, 1, 1, 1, 1, 1, 1, ...}
\CommentTok{#> $ hour       <int> 1, 2, 3, 4, 5, 6, 7, 8, 9, 10, 11, 13, 14, 15, 16, ...}
\CommentTok{#> $ temp       <dbl> 39.02, 39.02, 39.02, 39.92, 39.02, 37.94, 39.02, 39...}
\CommentTok{#> $ dewp       <dbl> 26.06, 26.96, 28.04, 28.04, 28.04, 28.04, 28.04, 28...}
\CommentTok{#> $ humid      <dbl> 59.37, 61.63, 64.43, 62.21, 64.43, 67.21, 64.43, 62...}
\CommentTok{#> $ wind_dir   <dbl> 270, 250, 240, 250, 260, 240, 240, 250, 260, 260, 2...}
\CommentTok{#> $ wind_speed <dbl> 10.35702, 8.05546, 11.50780, 12.65858, 12.65858, 11...}
\CommentTok{#> $ wind_gust  <dbl> NA, NA, NA, NA, NA, NA, NA, NA, NA, NA, NA, NA, NA,...}
\CommentTok{#> $ precip     <dbl> 0, 0, 0, 0, 0, 0, 0, 0, 0, 0, 0, 0, 0, 0, 0, 0, 0, ...}
\CommentTok{#> $ pressure   <dbl> 1012.0, 1012.3, 1012.5, 1012.2, 1011.9, 1012.4, 101...}
\CommentTok{#> $ visib      <dbl> 10, 10, 10, 10, 10, 10, 10, 10, 10, 10, 10, 10, 10,...}
\CommentTok{#> $ time_hour  <dttm> 2013-01-01 01:00:00, 2013-01-01 02:00:00, 2013-01-...}

\NormalTok{weather2 <-}\StringTok{ }\NormalTok{weather }\OperatorTok\StringTok{ }
\StringTok{  }\KeywordTok{mutate}\NormalTok{(}\DataTypeTok{date_time =} \KeywordTok{round_date}\NormalTok{(time_hour, }\StringTok{"hour"}\NormalTok{),}
         \DataTypeTok{day_rounded =} \KeywordTok{round_date}\NormalTok{(time_hour, }\StringTok{"day"}\NormalTok{)) }\OperatorTok\StringTok{ }
\StringTok{  }\KeywordTok{group_by}\NormalTok{(day_rounded) }\OperatorTok\StringTok{ }
\StringTok{  }\KeywordTok{summarise_at}\NormalTok{(}\KeywordTok{vars}\NormalTok{(temp, humid, wind_speed, precip, visib), median, }\DataTypeTok{na.rm =} \OtherTok{TRUE}\NormalTok{)}
\end{Highlighting}
\end{Shaded}

\begin{Shaded}
\begin{Highlighting}[]
\NormalTok{flights5 <-}\StringTok{ }\NormalTok{flights4 }\OperatorTok\StringTok{ }
\StringTok{  }\KeywordTok{inner_join}\NormalTok{(weather2, }\DataTypeTok{by =} \KeywordTok{c}\NormalTok{(}\StringTok{"day_rounded"}\NormalTok{ =}\StringTok{ "day_rounded"}\NormalTok{))}
\end{Highlighting}
\end{Shaded}

Wie ist die Korrelation der Wetterdaten mit der Verspätung?

\begin{Shaded}
\begin{Highlighting}[]
\NormalTok{flights5 }\OperatorTok\StringTok{ }
\StringTok{  }\KeywordTok{select}\NormalTok{(temp, humid, wind_speed, precip, visib, arr_delay) }\OperatorTok\StringTok{ }
\StringTok{  }\KeywordTok{correlate}\NormalTok{() }\OperatorTok\StringTok{ }
\StringTok{  }\KeywordTok{focus}\NormalTok{(arr_delay)}
\CommentTok{#> # A tibble: 5 x 2}
\CommentTok{#>   rowname    arr_delay}
\CommentTok{#>   <chr>          <dbl>}
\CommentTok{#> 1 temp          0.0216}
\CommentTok{#> 2 humid         0.150 }
\CommentTok{#> 3 wind_speed    0.0651}
\CommentTok{#> 4 precip        0.0465}
\CommentTok{#> 5 visib        -0.0702}
\end{Highlighting}
\end{Shaded}

Gut, etwas Zusammenhang mit Luftfeuchtigkeit (\texttt{humid}), aber
ansonsten nicht viel zu sehen. Apropos sehen: Schlechte Sicht geht mit
\emph{weniger} Verspätung einher (?).

\hypertarget{modellierung}{%
\subsection{Modellierung}\label{modellierung}}

\hypertarget{datensatz-bereinigen}{%
\subsubsection{Datensatz bereinigen}\label{datensatz-bereinigen}}

\hypertarget{variablen-ohne-varianz}{%
\paragraph{Variablen ohne Varianz}\label{variablen-ohne-varianz}}

Variablen ohne Varianz sind wertlos für die Vorhersage, also entfernen
wir sie. \texttt{caret} bietet dazu \texttt{nearZeroVar}. Natürlich kann
man sich auch die Daten mit bloßem Auge ansehen, dann fällt auf, dass
\texttt{year} den konstanten Wert 2013 aufweist.

\begin{Shaded}
\begin{Highlighting}[]
\NormalTok{flights6 <-}\StringTok{ }\NormalTok{flights5 }\OperatorTok\StringTok{ }
\StringTok{  }\KeywordTok{select}\NormalTok{(}\OperatorTok{-}\NormalTok{year)}
\end{Highlighting}
\end{Shaded}

\hypertarget{fehlende-werte}{%
\paragraph{Fehlende Werte}\label{fehlende-werte}}

Probieren wir die rabiate Methode:

\begin{Shaded}
\begin{Highlighting}[]
\NormalTok{flights7 <-}\StringTok{ }\NormalTok{flights6 }\OperatorTok\StringTok{ }
\StringTok{  }\KeywordTok{drop_na}\NormalTok{()}
\end{Highlighting}
\end{Shaded}

Wie viel Prozent der Fälle haben wir verloren?

\begin{Shaded}
\begin{Highlighting}[]
\KeywordTok{nrow}\NormalTok{(flights7)}\OperatorTok{/}\KeywordTok{nrow}\NormalTok{(flights6)}
\CommentTok{#> [1] 0.9719892}
\end{Highlighting}
\end{Shaded}

Etwa 3\%, das verschmerzen wir.

\hypertarget{z-skalieren}{%
\paragraph{Z-Skalieren}\label{z-skalieren}}

Für viele Algorithmen ist es nötig (z.B. neuronale Netze), die
Prädiktoren vorab zu standardisieren hinsichtlich Mittelwert und
Streuung (z-Transformation). Das kann man z.B. so erreichen (via Paket
\texttt{sjmisc}):

\begin{Shaded}
\begin{Highlighting}[]
\NormalTok{flights7a <-}\StringTok{ }\KeywordTok{std}\NormalTok{(flights7, }\DataTypeTok{suffix =} \StringTok{""}\NormalTok{) }
\end{Highlighting}
\end{Shaded}

\hypertarget{kreuzvalidierungsmethode}{%
\subsubsection{Kreuzvalidierungsmethode}\label{kreuzvalidierungsmethode}}

Um Überanpassung zu vermeiden, verwenden wir eine
5-fach-Kreuzvalidierung.

\begin{Shaded}
\begin{Highlighting}[]
\NormalTok{my_crossval <-}\StringTok{ }\KeywordTok{trainControl}\NormalTok{(}\DataTypeTok{method =} \StringTok{"cv"}\NormalTok{,}
                            \DataTypeTok{number =} \DecValTok{5}\NormalTok{,}
                            \DataTypeTok{allowParallel =} \OtherTok{TRUE}\NormalTok{,}
                            \DataTypeTok{verboseIter =} \OtherTok{FALSE}\NormalTok{)}
\end{Highlighting}
\end{Shaded}

\texttt{allowParallel} erlaubt die Verwendung mehrerer Rechenkerne,
sofern initialisiert:

\begin{Shaded}
\begin{Highlighting}[]
\NormalTok{doMC}\OperatorTok{::}\KeywordTok{registerDoMC}\NormalTok{(}\DataTypeTok{cores =} \DecValTok{2}\NormalTok{)}
\end{Highlighting}
\end{Shaded}

Achtung: Verdoppeln wir die Anzahl der Kerne, verdoppeln wir damit auch
die Menge des benötigten Speichers.

\hypertarget{datensatz-reduzieren}{%
\subsubsection{Datensatz reduzieren}\label{datensatz-reduzieren}}

Große Datensätze bringen einen Rechner leicht aus der Ruhe. Besonders
kategoriale Variablen mit vielen Stufen sind schwierig, da sie (manuell
oder je nach Funktion automatisch) in Dummy-Variablen umgewandelt werden
müssen.

Begrenzen wir uns daher auf metrische Variablen.

\begin{Shaded}
\begin{Highlighting}[]
\NormalTok{flights8 <-}\StringTok{ }\NormalTok{flights7a }\OperatorTok\StringTok{ }
\StringTok{  }\KeywordTok{select_if}\NormalTok{(is.numeric)}
\end{Highlighting}
\end{Shaded}

Außerdem dürfen wir nicht vergessen, die andere Verspätungsvariable zu
entferne (\texttt{dep\_delay}).

\begin{Shaded}
\begin{Highlighting}[]
\NormalTok{flights9 <-}\StringTok{ }\NormalTok{flights8 }\OperatorTok\StringTok{ }
\StringTok{  }\KeywordTok{select}\NormalTok{(}\OperatorTok{-}\NormalTok{dep_delay)}
\end{Highlighting}
\end{Shaded}

\hypertarget{redundante-variablen}{%
\subsubsection{Redundante Variablen}\label{redundante-variablen}}

Haben wir noch redundante Variablen?

\begin{Shaded}
\begin{Highlighting}[]
\KeywordTok{findLinearCombos}\NormalTok{(flights9)}
\CommentTok{#> $linearCombos}
\CommentTok{#> $linearCombos[[1]]}
\CommentTok{#> [1] 12  4 11}
\CommentTok{#> }
\CommentTok{#> $linearCombos[[2]]}
\CommentTok{#> [1] 14 13}
\CommentTok{#> }
\CommentTok{#> }
\CommentTok{#> $remove}
\CommentTok{#> [1] 12 14}
\end{Highlighting}
\end{Shaded}

Ja!

Entfernen wir sie:

\begin{Shaded}
\begin{Highlighting}[]
\NormalTok{flights9a <-}\StringTok{ }\NormalTok{flights9 }\OperatorTok\StringTok{ }
\StringTok{  }\KeywordTok{select}\NormalTok{(}\OperatorTok{-}\KeywordTok{c}\NormalTok{(}\DecValTok{12}\NormalTok{,}\DecValTok{14}\NormalTok{))}
\end{Highlighting}
\end{Shaded}

\hypertarget{datensatz-aufteilen}{%
\subsubsection{Datensatz aufteilen}\label{datensatz-aufteilen}}

Teilen wir den Datensatz zu 80\% in einen Übungsteil bzw. zu 20\% in
einen Testdatensatz auf.

\begin{Shaded}
\begin{Highlighting}[]
\NormalTok{n_uebung <-}\StringTok{ }\KeywordTok{round}\NormalTok{(.}\DecValTok{8} \OperatorTok{*}\StringTok{ }\KeywordTok{nrow}\NormalTok{(flights9a), }\DataTypeTok{digits =} \DecValTok{0}\NormalTok{)}

\NormalTok{uebung_index <-}\StringTok{ }\KeywordTok{sample}\NormalTok{(}\DecValTok{1}\OperatorTok{:}\KeywordTok{nrow}\NormalTok{(flights9a), }\DataTypeTok{size =}\NormalTok{ n_uebung)}

\NormalTok{uebung_df <-}\StringTok{ }\KeywordTok{filter}\NormalTok{(flights9a, }\KeywordTok{row_number}\NormalTok{() }\OperatorTok\StringTok{ }\NormalTok{uebung_index)}
\NormalTok{test_df <-}\StringTok{ }\KeywordTok{filter}\NormalTok{(flights9a, }\OperatorTok{!}\NormalTok{(}\KeywordTok{row_number}\NormalTok{() }\OperatorTok\StringTok{ }\NormalTok{uebung_index))}
\end{Highlighting}
\end{Shaded}

Die Gesamtfallzahl muss der Summe aus Übungs- und Test-Datensatz
entsprechen:

\begin{Shaded}
\begin{Highlighting}[]
\NormalTok{(}\KeywordTok{nrow}\NormalTok{(uebung_df) }\OperatorTok{+}\StringTok{ }\KeywordTok{nrow}\NormalTok{(test_df)) }\OperatorTok{==}\StringTok{ }\KeywordTok{nrow}\NormalTok{(flights9a)}
\CommentTok{#> [1] TRUE}
\end{Highlighting}
\end{Shaded}

Passt.

\hypertarget{modell-1---regression}{%
\subsubsection{Modell 1 - Regression}\label{modell-1---regression}}

Beginnen wir mit einer einfachen Regression (ohne Interaktionen,
Polynome, etc.).\footnote{vgl.
  \url{https://topepo.github.io/caret/train-models-by-tag.html\#linear-regression}}

Eine Regression hat keine Tuningparameter.

\begin{Shaded}
\begin{Highlighting}[]
\NormalTok{start <-}\StringTok{ }\KeywordTok{Sys.time}\NormalTok{()}
\NormalTok{lm_fit1 <-}\StringTok{ }\KeywordTok{train}\NormalTok{(arr_delay }\OperatorTok{~}\StringTok{ }\NormalTok{.,}
                 \DataTypeTok{data =}\NormalTok{ uebung_df,}
                 \DataTypeTok{method =} \StringTok{"lm"}\NormalTok{,}
                 \DataTypeTok{trControl =}\NormalTok{ my_crossval)}
\NormalTok{end <-}\StringTok{ }\KeywordTok{Sys.time}\NormalTok{()}

\NormalTok{(time_taken_lm1 <-}\StringTok{ }\NormalTok{end }\OperatorTok{-}\StringTok{ }\NormalTok{start)}
\CommentTok{#> Time difference of 25.82482 secs}


\CommentTok{#saveRDS(lm_fit1, file = "lm_fit1.rds")}
\end{Highlighting}
\end{Shaded}

Ohne die Begrenzung auf numerische Variablen hat meine Maschine (16GB
Speicher) einen Asthmaanfall bekommen und ist
steckengeblieben.\footnote{vgl.
  \url{https://stackoverflow.com/questions/51248293/error-vector-memory-exhausted-limit-reached-r-3-5-0-macos?rq=1}}

Das erzeugte Modell hatte in der Datei eine Größe von ca. 360MB.

Die Koeffizienten des Modells lassen sich auf übliche Weise bestimmen:

\begin{Shaded}
\begin{Highlighting}[]
\KeywordTok{summary}\NormalTok{(lm_fit1)}
\CommentTok{#> }
\CommentTok{#> Call:}
\CommentTok{#> lm(formula = .outcome ~ ., data = dat)}
\CommentTok{#> }
\CommentTok{#> Residuals:}
\CommentTok{#>     Min      1Q  Median      3Q     Max }
\CommentTok{#> -2.1331 -0.4664 -0.1819  0.1809 29.4201 }
\CommentTok{#> }
\CommentTok{#> Coefficients:}
\CommentTok{#>                  Estimate Std. Error  t value Pr(>|t|)    }
\CommentTok{#> (Intercept)     0.0006767  0.0017679    0.383  0.70189    }
\CommentTok{#> month           8.5536830  0.3026523   28.262  < 2e-16 ***}
\CommentTok{#> day             0.7325911  0.0255662   28.655  < 2e-16 ***}
\CommentTok{#> dep_time        0.7870591  0.0061406  128.173  < 2e-16 ***}
\CommentTok{#> sched_dep_time -0.5949524  0.0435803  -13.652  < 2e-16 ***}
\CommentTok{#> arr_time       -0.2712591  0.0029136  -93.102  < 2e-16 ***}
\CommentTok{#> sched_arr_time  0.0960639  0.0036036   26.657  < 2e-16 ***}
\CommentTok{#> flight          0.0371456  0.0020263   18.332  < 2e-16 ***}
\CommentTok{#> air_time        1.6411560  0.0141639  115.869  < 2e-16 ***}
\CommentTok{#> distance       -1.6623510  0.0142112 -116.975  < 2e-16 ***}
\CommentTok{#> hour            0.1226310  0.0430038    2.852  0.00435 ** }
\CommentTok{#> dayinyear      -8.9888118  0.3186638  -28.208  < 2e-16 ***}
\CommentTok{#> week            0.3882411  0.0910250    4.265    2e-05 ***}
\CommentTok{#> temp            0.0742258  0.0023826   31.154  < 2e-16 ***}
\CommentTok{#> humid           0.1524809  0.0022081   69.055  < 2e-16 ***}
\CommentTok{#> wind_speed      0.0971761  0.0019647   49.461  < 2e-16 ***}
\CommentTok{#> precip         -0.0012373  0.0019184   -0.645  0.51894    }
\CommentTok{#> visib           0.0063319  0.0022010    2.877  0.00402 ** }
\CommentTok{#> ---}
\CommentTok{#> Signif. codes:  0 '***' 0.001 '**' 0.01 '*' 0.05 '.' 0.1 ' ' 1}
\CommentTok{#> }
\CommentTok{#> Residual standard error: 0.9041 on 261512 degrees of freedom}
\CommentTok{#> Multiple R-squared:  0.1867, Adjusted R-squared:  0.1867 }
\CommentTok{#> F-statistic:  3532 on 17 and 261512 DF,  p-value: < 2.2e-16}
\end{Highlighting}
\end{Shaded}

Die Prädiktorenrelevanz kann man über \texttt{varImp()} abfragen.

\begin{Shaded}
\begin{Highlighting}[]
\KeywordTok{varImp}\NormalTok{(lm_fit1)}
\CommentTok{#> lm variable importance}
\CommentTok{#> }
\CommentTok{#>                Overall}
\CommentTok{#> dep_time       100.000}
\CommentTok{#> distance        91.219}
\CommentTok{#> air_time        90.352}
\CommentTok{#> arr_time        72.499}
\CommentTok{#> humid           53.643}
\CommentTok{#> wind_speed      38.279}
\CommentTok{#> temp            23.923}
\CommentTok{#> day             21.964}
\CommentTok{#> month           21.656}
\CommentTok{#> dayinyear       21.613}
\CommentTok{#> sched_arr_time  20.397}
\CommentTok{#> flight          13.869}
\CommentTok{#> sched_dep_time  10.199}
\CommentTok{#> week             2.839}
\CommentTok{#> visib            1.750}
\CommentTok{#> hour             1.730}
\CommentTok{#> precip           0.000}
\end{Highlighting}
\end{Shaded}

\hypertarget{modell-2---random-forest}{%
\subsubsection{Modell 2 - Random
Forest}\label{modell-2---random-forest}}

Im Gegensatz zur Regression gibt es bei Random-Forest-Modellen
Tuningparameter, und zwar die Anzahl der Variablen pro Baum, hier mit
\texttt{.mtry} bezeichnet. Eine Faustregel für diesen Parameter ist
\(\sqrt(k)\), hier also etwa oder 6.

\begin{Shaded}
\begin{Highlighting}[]
\NormalTok{rf_grid <-}\StringTok{ }\KeywordTok{data.frame}\NormalTok{(}
  \DataTypeTok{.mtry =} \KeywordTok{c}\NormalTok{(}\DecValTok{4}\NormalTok{, }\DecValTok{5}\NormalTok{, }\DecValTok{6}\NormalTok{, }\DecValTok{7}\NormalTok{),}
  \DataTypeTok{.splitrule =} \StringTok{"variance"}\NormalTok{,}
  \DataTypeTok{.min.node.size =} \DecValTok{5}\NormalTok{)}

\NormalTok{rf_grid}
\CommentTok{#>   .mtry .splitrule .min.node.size}
\CommentTok{#> 1     4   variance              5}
\CommentTok{#> 2     5   variance              5}
\CommentTok{#> 3     6   variance              5}
\CommentTok{#> 4     7   variance              5}
\end{Highlighting}
\end{Shaded}

Um Zeit zu sparen, verringern wir die Stichprobengröße auf 1000:

\begin{Shaded}
\begin{Highlighting}[]
\NormalTok{uebung_df_small <-}\StringTok{ }\KeywordTok{sample_n}\NormalTok{(uebung_df, }\DataTypeTok{size =} \DecValTok{1000}\NormalTok{)}
\end{Highlighting}
\end{Shaded}

Dann berechnen wir das Modell; gibt man keine Hinweise auf Variation von
Tuningparametern, so wählt die Funktion Standardwerte.

\begin{Shaded}
\begin{Highlighting}[]
\NormalTok{start <-}\StringTok{ }\KeywordTok{Sys.time}\NormalTok{()}
\NormalTok{rf_fit1 <-}\StringTok{ }\KeywordTok{train}\NormalTok{(arr_delay }\OperatorTok{~}\StringTok{ }\NormalTok{.,}
                 \DataTypeTok{data =}\NormalTok{ uebung_df_small,}
                 \DataTypeTok{method =} \StringTok{"ranger"}\NormalTok{,}
                 \DataTypeTok{trControl =}\NormalTok{ my_crossval)}
\NormalTok{end <-}\StringTok{ }\KeywordTok{Sys.time}\NormalTok{()}

\NormalTok{(time_taken <-}\StringTok{ }\NormalTok{end }\OperatorTok{-}\StringTok{ }\NormalTok{start)}
\CommentTok{#> Time difference of 23.08801 secs}

\CommentTok{#saveRDS(rf_fit1, file = "lm_fit1.rds")}
\CommentTok{#readRDS("lm_fit1.rds")}
\end{Highlighting}
\end{Shaded}

Einen Überblick über das berechnete Modell kann man sich so ausgeben
lassen:

\begin{Shaded}
\begin{Highlighting}[]
\NormalTok{rf_fit1}
\CommentTok{#> Random Forest }
\CommentTok{#> }
\CommentTok{#> 1000 samples}
\CommentTok{#>   17 predictor}
\CommentTok{#> }
\CommentTok{#> No pre-processing}
\CommentTok{#> Resampling: Cross-Validated (5 fold) }
\CommentTok{#> Summary of sample sizes: 800, 801, 801, 799, 799 }
\CommentTok{#> Resampling results across tuning parameters:}
\CommentTok{#> }
\CommentTok{#>   mtry  splitrule   RMSE       Rsquared   MAE      }
\CommentTok{#>    2    variance    0.8253691  0.4129177  0.5272027}
\CommentTok{#>    2    extratrees  0.8590837  0.3932634  0.5362638}
\CommentTok{#>    9    variance    0.6857243  0.5992653  0.4387015}
\CommentTok{#>    9    extratrees  0.7129957  0.5855666  0.4599433}
\CommentTok{#>   17    variance    0.6646842  0.6178340  0.4125599}
\CommentTok{#>   17    extratrees  0.6652444  0.6377894  0.4281044}
\CommentTok{#> }
\CommentTok{#> Tuning parameter 'min.node.size' was held constant at a value of 5}
\CommentTok{#> RMSE was used to select the optimal model using the smallest value.}
\CommentTok{#> The final values used for the model were mtry = 17, splitrule =}
\CommentTok{#>  variance and min.node.size = 5.}
\end{Highlighting}
\end{Shaded}

Im resultierenden Objekt sind eine Vielzahl von Informationen zu finden.
So kann man sich den Modellkandidaten mit den besten Werten ausgeben
lassen:

\begin{Shaded}
\begin{Highlighting}[]
\NormalTok{rf_fit1}\OperatorTok{$}\NormalTok{bestTune}
\CommentTok{#>   mtry splitrule min.node.size}
\CommentTok{#> 5   17  variance             5}
\end{Highlighting}
\end{Shaded}

Aber was ist das Kriteriem, das optimiert wird?

\begin{Shaded}
\begin{Highlighting}[]
\NormalTok{rf_fit1}\OperatorTok{$}\NormalTok{metric}
\CommentTok{#> [1] "RMSE"}
\end{Highlighting}
\end{Shaded}

Es wird nach dem \emph{Root Mean Square Error} optimiert.

Weitere Infos zum Algorithmus bekommt man z.B. so:

\begin{Shaded}
\begin{Highlighting}[]
\KeywordTok{modelLookup}\NormalTok{(}\StringTok{"ranger"}\NormalTok{)}
\CommentTok{#>    model     parameter                         label forReg forClass}
\CommentTok{#> 1 ranger          mtry #Randomly Selected Predictors   TRUE     TRUE}
\CommentTok{#> 2 ranger     splitrule                Splitting Rule   TRUE     TRUE}
\CommentTok{#> 3 ranger min.node.size             Minimal Node Size   TRUE     TRUE}
\CommentTok{#>   probModel}
\CommentTok{#> 1      TRUE}
\CommentTok{#> 2      TRUE}
\CommentTok{#> 3      TRUE}
\end{Highlighting}
\end{Shaded}

Wir sehen, dass das Modell drei Tuningparameter hat (wobei
\texttt{caret} den Parameter \texttt{min.node.size} konstant hielt).

\begin{Shaded}
\begin{Highlighting}[]
\KeywordTok{plot}\NormalTok{(rf_fit1)}
\end{Highlighting}
\end{Shaded}

\begin{center}\includegraphics[width=0.7\linewidth]{Fallstudie-nyc-flights_files/figure-latex/unnamed-chunk-37-1} \end{center}

Möchte man eine bestimmte Anzahl an Kandidatenmodelle prüfen lassen, so
kann man das mit \texttt{tuneLength} tun:

\begin{Shaded}
\begin{Highlighting}[]
\NormalTok{start <-}\StringTok{ }\KeywordTok{Sys.time}\NormalTok{()}
\NormalTok{rf_fit2 <-}\StringTok{ }\KeywordTok{train}\NormalTok{(arr_delay }\OperatorTok{~}\StringTok{ }\NormalTok{.,}
                 \DataTypeTok{data =}\NormalTok{ uebung_df_small,}
                 \DataTypeTok{method =} \StringTok{"ranger"}\NormalTok{,}
                 \DataTypeTok{trControl =}\NormalTok{ my_crossval,}
                 \DataTypeTok{tuneLength =} \DecValTok{4}\NormalTok{)}
\NormalTok{end <-}\StringTok{ }\KeywordTok{Sys.time}\NormalTok{()}


\NormalTok{(time_taken <-}\StringTok{ }\NormalTok{end }\OperatorTok{-}\StringTok{ }\NormalTok{start)}
\CommentTok{#> Time difference of 36.44905 secs}

\KeywordTok{saveRDS}\NormalTok{(rf_fit2, }\DataTypeTok{file =} \StringTok{"lm_fit2.rds"}\NormalTok{)}
\end{Highlighting}
\end{Shaded}

\begin{Shaded}
\begin{Highlighting}[]
\NormalTok{rf_fit2}
\CommentTok{#> Random Forest }
\CommentTok{#> }
\CommentTok{#> 1000 samples}
\CommentTok{#>   17 predictor}
\CommentTok{#> }
\CommentTok{#> No pre-processing}
\CommentTok{#> Resampling: Cross-Validated (5 fold) }
\CommentTok{#> Summary of sample sizes: 799, 800, 800, 801, 800 }
\CommentTok{#> Resampling results across tuning parameters:}
\CommentTok{#> }
\CommentTok{#>   mtry  splitrule   RMSE       Rsquared   MAE      }
\CommentTok{#>    2    variance    0.8203238  0.4175518  0.5321533}
\CommentTok{#>    2    extratrees  0.8525036  0.4032347  0.5362929}
\CommentTok{#>    7    variance    0.7002828  0.5966789  0.4562868}
\CommentTok{#>    7    extratrees  0.7199919  0.5847068  0.4712169}
\CommentTok{#>   12    variance    0.6630210  0.6427126  0.4264476}
\CommentTok{#>   12    extratrees  0.6755618  0.6339950  0.4431332}
\CommentTok{#>   17    variance    0.6575474  0.6444369  0.4155007}
\CommentTok{#>   17    extratrees  0.6563739  0.6510467  0.4311083}
\CommentTok{#> }
\CommentTok{#> Tuning parameter 'min.node.size' was held constant at a value of 5}
\CommentTok{#> RMSE was used to select the optimal model using the smallest value.}
\CommentTok{#> The final values used for the model were mtry = 17, splitrule =}
\CommentTok{#>  extratrees and min.node.size = 5.}
\end{Highlighting}
\end{Shaded}

Mit \texttt{tuneGrid} kann man die Werte der Modellkandidaten genau
einstellen.

\begin{Shaded}
\begin{Highlighting}[]
\NormalTok{start <-}\StringTok{ }\KeywordTok{Sys.time}\NormalTok{()}
\NormalTok{rf_fit3 <-}\StringTok{ }\KeywordTok{train}\NormalTok{(arr_delay }\OperatorTok{~}\StringTok{ }\NormalTok{.,}
                 \DataTypeTok{data =}\NormalTok{ uebung_df_small,}
                 \DataTypeTok{method =} \StringTok{"ranger"}\NormalTok{,}
                 \DataTypeTok{trControl =}\NormalTok{ my_crossval,}
                 \DataTypeTok{tuneGrid =}\NormalTok{ rf_grid)}
\NormalTok{end <-}\StringTok{ }\KeywordTok{Sys.time}\NormalTok{()}


\NormalTok{(time_taken <-}\StringTok{ }\NormalTok{end }\OperatorTok{-}\StringTok{ }\NormalTok{start)}
\CommentTok{#> Time difference of 8.39345 secs}

\CommentTok{# saveRDS(rf_fit3, file = "lm_fit3.rds")}
\end{Highlighting}
\end{Shaded}

\begin{Shaded}
\begin{Highlighting}[]
\NormalTok{rf_fit3}
\CommentTok{#> Random Forest }
\CommentTok{#> }
\CommentTok{#> 1000 samples}
\CommentTok{#>   17 predictor}
\CommentTok{#> }
\CommentTok{#> No pre-processing}
\CommentTok{#> Resampling: Cross-Validated (5 fold) }
\CommentTok{#> Summary of sample sizes: 800, 800, 801, 799, 800 }
\CommentTok{#> Resampling results across tuning parameters:}
\CommentTok{#> }
\CommentTok{#>   mtry  RMSE       Rsquared   MAE      }
\CommentTok{#>   4     0.7553441  0.5083364  0.4943232}
\CommentTok{#>   5     0.7254933  0.5542789  0.4785002}
\CommentTok{#>   6     0.7045373  0.5814886  0.4632984}
\CommentTok{#>   7     0.6926108  0.6007891  0.4535480}
\CommentTok{#> }
\CommentTok{#> Tuning parameter 'splitrule' was held constant at a value of}
\CommentTok{#>  variance}
\CommentTok{#> Tuning parameter 'min.node.size' was held constant at a value}
\CommentTok{#>  of 5}
\CommentTok{#> RMSE was used to select the optimal model using the smallest value.}
\CommentTok{#> The final values used for the model were mtry = 7, splitrule =}
\CommentTok{#>  variance and min.node.size = 5.}
\end{Highlighting}
\end{Shaded}

\hypertarget{modell-3---neuronales-netz}{%
\subsubsection{Modell 3 - Neuronales
Netz}\label{modell-3---neuronales-netz}}

Neuronale Netze benötigen große Datensätze, daher ist unser kleiner
Datensatz mit \(n=1000\) sicher zu klein (gerade in Anbetracht zur Zahl
der Features). Aus Vergleichsbarkeitsgründen und um die Rechenkosten zu
schätzen, bietet es sich aber an, zunächst mit einem kleinen Datensatz
zu arbeiten:

\begin{Shaded}
\begin{Highlighting}[]
\NormalTok{start <-}\StringTok{ }\KeywordTok{Sys.time}\NormalTok{()}
\NormalTok{nn_fit1 <-}\StringTok{ }\KeywordTok{train}\NormalTok{(arr_delay }\OperatorTok{~}\StringTok{ }\NormalTok{.,}
                 \DataTypeTok{data =}\NormalTok{ uebung_df_small,}
                 \DataTypeTok{method =} \StringTok{"nnet"}\NormalTok{,}
                 \DataTypeTok{trControl =}\NormalTok{ my_crossval,}
                 \DataTypeTok{linout =} \OtherTok{TRUE}\NormalTok{)}
\CommentTok{#> # weights:  96}
\CommentTok{#> initial  value 1293.844765 }
\CommentTok{#> iter  10 value 810.417182}
\CommentTok{#> iter  20 value 599.392707}
\CommentTok{#> iter  30 value 449.445228}
\CommentTok{#> iter  40 value 364.643107}
\CommentTok{#> iter  50 value 289.611282}
\CommentTok{#> iter  60 value 238.267314}
\CommentTok{#> iter  70 value 209.798648}
\CommentTok{#> iter  80 value 175.217991}
\CommentTok{#> iter  90 value 162.966931}
\CommentTok{#> iter 100 value 152.345308}
\CommentTok{#> final  value 152.345308 }
\CommentTok{#> stopped after 100 iterations}
\NormalTok{end <-}\StringTok{ }\KeywordTok{Sys.time}\NormalTok{()}


\NormalTok{(time_taken <-}\StringTok{ }\NormalTok{end }\OperatorTok{-}\StringTok{ }\NormalTok{start)}
\CommentTok{#> Time difference of 4.283968 secs}

\KeywordTok{saveRDS}\NormalTok{(nn_fit1, }\DataTypeTok{file =} \StringTok{"nn_fit1.rds"}\NormalTok{)}
\end{Highlighting}
\end{Shaded}

Der Parameter \texttt{linout\ =\ TRUE} verhindert eine
Aktivierungsfunktion, die den Wertebereich auf {[}0,1{]} beschränken
würde.

Das ging schnell. Vergrößen wir den Datensatz:

\begin{Shaded}
\begin{Highlighting}[]
\NormalTok{start <-}\StringTok{ }\KeywordTok{Sys.time}\NormalTok{()}
\NormalTok{nn_fit2 <-}\StringTok{ }\KeywordTok{train}\NormalTok{(arr_delay }\OperatorTok{~}\StringTok{ }\NormalTok{.,}
                 \DataTypeTok{data =}\NormalTok{ uebung_df,}
                 \DataTypeTok{method =} \StringTok{"nnet"}\NormalTok{,}
                 \DataTypeTok{trControl =}\NormalTok{ my_crossval,}
                 \DataTypeTok{linout =} \OtherTok{TRUE}\NormalTok{)}
\CommentTok{#> # weights:  96}
\CommentTok{#> initial  value 282715.082894 }
\CommentTok{#> iter  10 value 231584.339546}
\CommentTok{#> iter  20 value 178470.397954}
\CommentTok{#> iter  30 value 153481.591084}
\CommentTok{#> iter  40 value 135576.240699}
\CommentTok{#> iter  50 value 116695.609467}
\CommentTok{#> iter  60 value 110859.369898}
\CommentTok{#> iter  70 value 103110.157901}
\CommentTok{#> iter  80 value 95147.150020}
\CommentTok{#> iter  90 value 85477.968554}
\CommentTok{#> iter 100 value 74358.691471}
\CommentTok{#> final  value 74358.691471 }
\CommentTok{#> stopped after 100 iterations}
\NormalTok{end <-}\StringTok{ }\KeywordTok{Sys.time}\NormalTok{()}


\NormalTok{(time_taken <-}\StringTok{ }\NormalTok{end }\OperatorTok{-}\StringTok{ }\NormalTok{start)}
\CommentTok{#> Time difference of 13.16125 mins}

\CommentTok{# saveRDS(nn_fit2, file = "nn_fit2.rds")}
\end{Highlighting}
\end{Shaded}

Es gibt verschiedene Implementierungen von neuronalen Netzen, die in
\texttt{caret} angesteuert werden können, z.B. \texttt{neuralnet}. Es
verfügt über 3 Modellparameter:

\begin{Shaded}
\begin{Highlighting}[]
\KeywordTok{modelLookup}\NormalTok{(}\StringTok{"neuralnet"}\NormalTok{)}
\CommentTok{#>       model parameter                    label forReg forClass probModel}
\CommentTok{#> 1 neuralnet    layer1 #Hidden Units in Layer 1   TRUE    FALSE     FALSE}
\CommentTok{#> 2 neuralnet    layer2 #Hidden Units in Layer 2   TRUE    FALSE     FALSE}
\CommentTok{#> 3 neuralnet    layer3 #Hidden Units in Layer 3   TRUE    FALSE     FALSE}
\end{Highlighting}
\end{Shaded}

\begin{Shaded}
\begin{Highlighting}[]
\KeywordTok{getModelInfo}\NormalTok{(}\StringTok{"neuralnet"}\NormalTok{)}
\CommentTok{#> $neuralnet}
\CommentTok{#> $neuralnet$label}
\CommentTok{#> [1] "Neural Network"}
\CommentTok{#> }
\CommentTok{#> $neuralnet$library}
\CommentTok{#> [1] "neuralnet"}
\CommentTok{#> }
\CommentTok{#> $neuralnet$loop}
\CommentTok{#> NULL}
\CommentTok{#> }
\CommentTok{#> $neuralnet$type}
\CommentTok{#> [1] "Regression"}
\CommentTok{#> }
\CommentTok{#> $neuralnet$parameters}
\CommentTok{#>   parameter   class                    label}
\CommentTok{#> 1    layer1 numeric #Hidden Units in Layer 1}
\CommentTok{#> 2    layer2 numeric #Hidden Units in Layer 2}
\CommentTok{#> 3    layer3 numeric #Hidden Units in Layer 3}
\CommentTok{#> }
\CommentTok{#> $neuralnet$grid}
\CommentTok{#> function(x, y, len = NULL, search = "grid") \{}
\CommentTok{#>                     if(search == "grid") \{}
\CommentTok{#>                       out <- expand.grid(layer1 = ((1:len) * 2) - 1, layer2 = 0, layer3 = 0)}
\CommentTok{#>                     \} else \{}
\CommentTok{#>                       out <- data.frame(layer1 = sample(2:20, replace = TRUE, size = len),}
\CommentTok{#>                                         layer2 = sample(c(0, 2:20), replace = TRUE, size = len),}
\CommentTok{#>                                         layer3 = sample(c(0, 2:20), replace = TRUE, size = len))}
\CommentTok{#>                     \}}
\CommentTok{#>                     out}
\CommentTok{#>                   \}}
\CommentTok{#> }
\CommentTok{#> $neuralnet$fit}
\CommentTok{#> function(x, y, wts, param, lev, last, classProbs, ...) \{}
\CommentTok{#>                     colNames <- colnames(x)}
\CommentTok{#>                     dat <- if(is.data.frame(x)) x else as.data.frame(x)}
\CommentTok{#>                     dat$.outcome <- y}
\CommentTok{#>                     form <- as.formula(paste(".outcome ~",paste(colNames, collapse = "+")))}
\CommentTok{#>                     if(param$layer1 == 0) stop("the first layer must have at least one hidden unit")}
\CommentTok{#>                     if(param$layer2 == 0 & param$layer2 > 0) stop("the second layer must have at least one hidden unit if a third layer is specified")}
\CommentTok{#>                     nodes <- c(param$layer1)}
\CommentTok{#>                     if(param$layer2 > 0) \{}
\CommentTok{#>                       nodes <- c(nodes, param$layer2)}
\CommentTok{#>                       if(param$layer3 > 0) nodes <- c(nodes, param$layer3)}
\CommentTok{#>                     \}}
\CommentTok{#>                     neuralnet::neuralnet(form, data = dat, hidden = nodes, ...)}
\CommentTok{#>                   \}}
\CommentTok{#> }
\CommentTok{#> $neuralnet$predict}
\CommentTok{#> function(modelFit, newdata, submodels = NULL) \{}
\CommentTok{#>                     newdata <- newdata[, modelFit$model.list$variables, drop = FALSE]}
\CommentTok{#>                     neuralnet::compute(modelFit, covariate = newdata)$net.result[,1]}
\CommentTok{#>                   \}}
\CommentTok{#> }
\CommentTok{#> $neuralnet$prob}
\CommentTok{#> NULL}
\CommentTok{#> }
\CommentTok{#> $neuralnet$tags}
\CommentTok{#> [1] "Neural Network"}
\CommentTok{#> }
\CommentTok{#> $neuralnet$sort}
\CommentTok{#> function(x) x[order(x$layer1, x$layer2, x$layer3),]}
\end{Highlighting}
\end{Shaded}

\hypertarget{vergleich-der-modellguten}{%
\subsection{Vergleich der Modellgüten}\label{vergleich-der-modellguten}}

Wie gut sagen die Modelle den Test-Datensatz vorher? Vergleichen wir die
Modelle.

\hypertarget{prognosen-fur-den-test-datensatz-berechnen}{%
\subsubsection{Prognosen für den Test-Datensatz
berechnen}\label{prognosen-fur-den-test-datensatz-berechnen}}

Erstellen wir uns einen Datensatz mit den Vorhersagen (und den
beobachteten Werten):

\begin{Shaded}
\begin{Highlighting}[]

\NormalTok{test_preds <-}\StringTok{ }\NormalTok{test_df }\OperatorTok\StringTok{ }
\StringTok{  }\KeywordTok{select}\NormalTok{(arr_delay) }\OperatorTok\StringTok{ }
\StringTok{  }\KeywordTok{mutate}\NormalTok{(}\DataTypeTok{lm1_pred =} \KeywordTok{predict}\NormalTok{(lm_fit1, }\DataTypeTok{newdata =}\NormalTok{ test_df))}

\NormalTok{test_preds <-}\StringTok{ }\NormalTok{test_df }\OperatorTok\StringTok{ }
\StringTok{  }\KeywordTok{select}\NormalTok{(arr_delay) }\OperatorTok\StringTok{ }
\StringTok{  }\KeywordTok{mutate}\NormalTok{(}\DataTypeTok{lm1_pred =} \KeywordTok{predict}\NormalTok{(lm_fit1, }\DataTypeTok{newdata =}\NormalTok{ test_df),}
         \DataTypeTok{rf1_pred =} \KeywordTok{predict}\NormalTok{(rf_fit2, }\DataTypeTok{newdata =}\NormalTok{ test_df),}
         \DataTypeTok{rf2_pred =} \KeywordTok{predict}\NormalTok{(rf_fit3, }\DataTypeTok{newdata =}\NormalTok{ test_df),}
         \DataTypeTok{nn1_pred =} \KeywordTok{predict}\NormalTok{(nn_fit1, }\DataTypeTok{newdata =}\NormalTok{ test_df),}
         \DataTypeTok{nn2_pred =} \KeywordTok{predict}\NormalTok{(nn_fit2, }\DataTypeTok{newdata =}\NormalTok{ test_df))}
\end{Highlighting}
\end{Shaded}

Jetzt lassen wir uns typische Kennzahlen der Modellgüte ausgeben:

\begin{Shaded}
\begin{Highlighting}[]
\KeywordTok{postResample}\NormalTok{(}\DataTypeTok{pred =}\NormalTok{ test_preds}\OperatorTok{$}\NormalTok{lm1_pred, }\DataTypeTok{obs =}\NormalTok{ test_preds}\OperatorTok{$}\NormalTok{arr_delay)}
\CommentTok{#>      RMSE  Rsquared       MAE }
\CommentTok{#> 0.8895674 0.1924120 0.5383087}
\end{Highlighting}
\end{Shaded}

Der Vektor der Modellnamen lautet:

\begin{Shaded}
\begin{Highlighting}[]
\NormalTok{model_names <-}\StringTok{ }\KeywordTok{names}\NormalTok{(test_preds)}
\end{Highlighting}
\end{Shaded}

Das wiederholen wir in einer Schleife für jedes Modell:

\begin{Shaded}
\begin{Highlighting}[]
\NormalTok{test_pred_df <-}\StringTok{ }\NormalTok{test_preds }\OperatorTok\StringTok{ }
\StringTok{  }\KeywordTok{map_df}\NormalTok{(}\OperatorTok{~}\StringTok{ }\KeywordTok{postResample}\NormalTok{(}\DataTypeTok{pred =}\NormalTok{ ., }\DataTypeTok{obs =}\NormalTok{ test_preds}\OperatorTok{$}\NormalTok{arr_delay)) }\OperatorTok\StringTok{ }
\StringTok{  }\KeywordTok{mutate}\NormalTok{(}\DataTypeTok{statistic =} \KeywordTok{c}\NormalTok{(}\StringTok{"RMSE"}\NormalTok{, }\StringTok{"Rsquared"}\NormalTok{, }\StringTok{"MAE"}\NormalTok{)) }\OperatorTok\StringTok{ }
\StringTok{  }\KeywordTok{select}\NormalTok{(statistic, }\KeywordTok{everything}\NormalTok{())}

\NormalTok{test_pred_df}
\CommentTok{#> # A tibble: 3 x 7}
\CommentTok{#>   statistic arr_delay lm1_pred rf1_pred rf2_pred nn1_pred nn2_pred}
\CommentTok{#>   <chr>         <dbl>    <dbl>    <dbl>    <dbl>    <dbl>    <dbl>}
\CommentTok{#> 1 RMSE              0    0.890    0.630    0.675    0.611    0.518}
\CommentTok{#> 2 Rsquared          1    0.192    0.633    0.567    0.644    0.727}
\CommentTok{#> 3 MAE               0    0.538    0.397    0.424    0.296    0.313}
\end{Highlighting}
\end{Shaded}

Formen wir diese Tabelle in Langform (Normalform um):

\begin{Shaded}
\begin{Highlighting}[]
\NormalTok{test_pred_df_t <-}\StringTok{ }\NormalTok{test_pred_df }\OperatorTok\StringTok{ }
\StringTok{  }\KeywordTok{gather}\NormalTok{(}\DataTypeTok{key =} \StringTok{"model_name"}\NormalTok{, }\DataTypeTok{value =} \StringTok{"value"}\NormalTok{, }\OperatorTok{-}\KeywordTok{c}\NormalTok{(statistic))}
\end{Highlighting}
\end{Shaded}

Eine andere Form wäre:

\begin{Shaded}
\begin{Highlighting}[]
\NormalTok{test_pred_df }\OperatorTok\StringTok{ }
\StringTok{  }\KeywordTok{gather}\NormalTok{(}\DataTypeTok{key =} \StringTok{"model_name"}\NormalTok{, }\DataTypeTok{value =} \StringTok{"value"}\NormalTok{, }\OperatorTok{-}\KeywordTok{c}\NormalTok{(statistic)) }\OperatorTok\StringTok{ }
\StringTok{  }\KeywordTok{spread}\NormalTok{(}\DataTypeTok{key =}\NormalTok{ statistic, }\DataTypeTok{value =}\NormalTok{ value)}
\CommentTok{#> # A tibble: 6 x 4}
\CommentTok{#>   model_name   MAE  RMSE Rsquared}
\CommentTok{#>   <chr>      <dbl> <dbl>    <dbl>}
\CommentTok{#> 1 arr_delay  0     0        1    }
\CommentTok{#> 2 lm1_pred   0.538 0.890    0.192}
\CommentTok{#> 3 nn1_pred   0.296 0.611    0.644}
\CommentTok{#> 4 nn2_pred   0.313 0.518    0.727}
\CommentTok{#> 5 rf1_pred   0.397 0.630    0.633}
\CommentTok{#> 6 rf2_pred   0.424 0.675    0.567}
\end{Highlighting}
\end{Shaded}

\hypertarget{bestes-modell-identifizieren}{%
\subsubsection{Bestes Modell
identifizieren}\label{bestes-modell-identifizieren}}

Der kleinste RMSE-Wert (nach dem Modell, dass als vorhergesagten Werte
die beobachteten nimmt, also einen RSME von Null hat):

\begin{Shaded}
\begin{Highlighting}[]
\NormalTok{test_pred_df_t }\OperatorTok\StringTok{ }
\StringTok{  }\KeywordTok{filter}\NormalTok{(statistic }\OperatorTok{==}\StringTok{ "RMSE"}\NormalTok{) }\OperatorTok\StringTok{ }
\StringTok{  }\KeywordTok{top_n}\NormalTok{(}\DecValTok{2}\NormalTok{, }\DataTypeTok{wt =} \OperatorTok{-}\NormalTok{value)}
\CommentTok{#> # A tibble: 2 x 3}
\CommentTok{#>   statistic model_name value}
\CommentTok{#>   <chr>     <chr>      <dbl>}
\CommentTok{#> 1 RMSE      arr_delay  0    }
\CommentTok{#> 2 RMSE      nn2_pred   0.518}
\end{Highlighting}
\end{Shaded}

Der größte R\^{}2-Wert:

\begin{Shaded}
\begin{Highlighting}[]
\NormalTok{test_pred_df_t }\OperatorTok\StringTok{ }
\StringTok{  }\KeywordTok{filter}\NormalTok{(statistic }\OperatorTok{==}\StringTok{ "Rsquared"}\NormalTok{) }\OperatorTok\StringTok{ }
\StringTok{  }\KeywordTok{top_n}\NormalTok{(}\DecValTok{2}\NormalTok{, }\DataTypeTok{wt =}\NormalTok{ value)}
\CommentTok{#> # A tibble: 2 x 3}
\CommentTok{#>   statistic model_name value}
\CommentTok{#>   <chr>     <chr>      <dbl>}
\CommentTok{#> 1 Rsquared  arr_delay  1    }
\CommentTok{#> 2 Rsquared  nn2_pred   0.727}
\end{Highlighting}
\end{Shaded}

\hypertarget{visualisieren}{%
\subsubsection{Visualisieren}\label{visualisieren}}

\begin{Shaded}
\begin{Highlighting}[]
\NormalTok{test_pred_df_t }\OperatorTok\StringTok{ }
\StringTok{  }\KeywordTok{group_by}\NormalTok{(statistic) }\OperatorTok\StringTok{ }
\StringTok{  }\KeywordTok{mutate}\NormalTok{(}\DataTypeTok{is_max =}\NormalTok{ value }\OperatorTok{==}\StringTok{ }\KeywordTok{max}\NormalTok{(value),}
         \DataTypeTok{is_min =}\NormalTok{ value }\OperatorTok{==}\StringTok{ }\KeywordTok{min}\NormalTok{(value)) }\OperatorTok\StringTok{ }
\StringTok{  }\KeywordTok{ggplot}\NormalTok{(}\KeywordTok{aes}\NormalTok{(}\DataTypeTok{y =}\NormalTok{ model_name, }\DataTypeTok{x =}\NormalTok{ value, }\DataTypeTok{color =}\NormalTok{ is_max, }\DataTypeTok{shape =}\NormalTok{ is_min)) }\OperatorTok{+}
\StringTok{  }\KeywordTok{geom_point}\NormalTok{(}\DataTypeTok{size =} \DecValTok{5}\NormalTok{) }\OperatorTok{+}
\StringTok{  }\KeywordTok{facet_wrap}\NormalTok{(}\OperatorTok{~}\NormalTok{statistic, }\DataTypeTok{scales =} \StringTok{"free_x"}\NormalTok{) }\OperatorTok{+}
\StringTok{  }\KeywordTok{theme}\NormalTok{(}\DataTypeTok{legend.position =} \StringTok{"bottom"}\NormalTok{)}
\end{Highlighting}
\end{Shaded}

\begin{center}\includegraphics[width=0.7\linewidth]{Fallstudie-nyc-flights_files/figure-latex/unnamed-chunk-50-1} \end{center}

Damit hat das neuronale Netz ``gewonnen''.

\hypertarget{fazit}{%
\section{Fazit}\label{fazit}}

Es darf nicht vergessen werden, dass wir nur einen Teil des Datensatzes
verwendet haben - schlicht aus Gründen der komputationalen
Kostensparung. Insofern sind die Modellgüten nur bedingt für bare Münze
zu nehmen. Diese Fallstudie hat nur einen Teil der Möglichkeiten einer
ernsthaften Modellierung aufgenommen, so dass die Ergebnisse schon aus
diesem Grund mit einem großen Gramm Salz zu betrachten sind. Außerdem
fanden nur relativ weniger Modell Eingang; es bleibt also offen, ob
nicht andere Modelle ``besser'' sind. Beim Wort ``besser'' muss man
immer im Kopf behalten, dass ``besser'' eine \emph{bedingte} Aussage
ist: \emph{besser} vor dem Hintergrund gewisser anderer Modelle,
gewisser Transformationen, gewisser Implementierungen, gewisser
Stichprobenmerkmale, gewisser Implementierungsspezifika und so weiter.


\end{document}
